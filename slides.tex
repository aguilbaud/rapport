\documentclass{beamer}
\usepackage[utf8]{inputenc}
\usepackage[francais]{babel}
\usepackage[utf8]{inputenc}  
\usepackage{listings}
\usepackage{graphicx}
\usepackage{color}
\usepackage{float}
\usepackage{algorithm}
\usepackage{algorithmic}
\usepackage{caption}

\usepackage{array}
\usepackage{colortbl}
\usepackage{amsfonts}
\usepackage{geometry}
\usepackage{setspace}
\usepackage{hyperref}
\usepackage{subcaption}
\usepackage{listings}
\usetheme{Warsaw}
\setbeamertemplate{footline}[frame number]

\PassOptionsToPackage{demo}{graphicx}




\author{Adrien Guilbaud}
\title{Simulation numérique directe de la combustion turbulente}
%\subtitle{Presentation Subtile}
%\institute{Université de Bordeaux}
%\date{\today}


%\titlegraphic{\includegraphics[width=2cm]{figures/logo_fac.jpg}\hspace*{4.75cm}~%
%   \includegraphics[width=2cm]{figures/logo_cerfacs.eps}
%}

\def\mathunderline#1{\color{red}\underline{{#1}}\color{black}}

\begin{document}
 

\begin{frame}
    \parbox[c]{-50cm}{\centering%
      \includegraphics[width=2cm]{figures/logo_fac.jpg}%
    }%
    \parbox[c]{19.5cm}{\centering%
      \includegraphics[width=2cm]{figures/logo_cerfacs.eps}
    }%
\maketitle

\centering
\footnotesize
\begin{tabular}{cc}
  Maître de stage & Enseignant responsable \\
  Mme.~Isabelle \textsc{D'ast} &   Mr.~Samuel \textsc{Thibault} \\
  \end{tabular}
\end{frame}

%
% INTRO
%

\section{Introduction}
\subsection{Présentation du Cerfacs}
\begin{frame}
  \begin{itemize}
  \item Centre Européen de Recherche et de Formation Avancée en Calcul Scientifique
  \item Airbus Group, Cnes, EDF, Météo France, Onera, Safran et Total
  
    \item Résolution de problèmes scientifiques par la résolution numérique liés :
	\begin{itemize}
  \item au climat
  \item à l'aéronautique
  \item au spatial
  \item à l'environnement
  \end{itemize}
  \end{itemize}


  
\end{frame}

%https://en.wikipedia.org/wiki/Discretization_of_Navier%E2%80%93Stokes_equations
\subsection{Mécanique des fluides numérique}
\begin{frame}
  \begin{block}{Mécanique des fluides}
    Etude du comportement des fluides lorsqu'ils sont en mouvement
  \end{block}
  
  \begin{itemize}
  \item Résolution des équations de Navier-Stokes
  \item Discrétisation de l'espace
  \end{itemize}
\end{frame}


\subsection{Simulation de la turbulence}
\begin{frame}
  \begin{itemize}
  \item \textit{DNS}: Direct Numerical Simulation
  \item \textit{LES}: Large Eddy Simulation
  \end{itemize}

  \begin{figure}[ht]
  \centering
  \begin{subfigure}[b]{0.5\textwidth}
    \centering
    \includegraphics[scale=0.25]{figures/DNS_Velocity_Field.png}
    \caption{\label{fig:dns} DNS}
  \end{subfigure}%
  ~
  \begin{subfigure}[b]{0.5\textwidth}
    \centering
    \includegraphics[scale=0.25]{figures/DNS_Filtered_Velocity_Field_Large.png}
    \caption{\label{fig:les} LES}
  \end{subfigure}
\end{figure}
\end{frame}




%
% Présentation stage
%
\section{Présentation du sujet}
\subsection{NTMIX\_CHEMKIN}
\begin{frame}
  \textit{NTMIX\_CHEMKIN}: solveur d'écoulements réactifs
  \begin{itemize}
  \item bidimensionnel
  \item approche DNS
  \item couplé avec \textit{CHEMKIN}
  \end{itemize}
  Intérêt en recherche fondamentale
  
\end{frame}


\subsection{Objectifs}
\begin{frame}

  \begin{block}{Objectif}
    Développer une version 3D et parallèle de \textit{NTMIX}
  \end{block}
  \begin{itemize}
  \item Modernisation du code
  \item Développement de la version tridimensionnelle
  \item Parallélisation la version 3D
  \item Étude des performances
  \end{itemize} 
\end{frame}


%
% Parallélisation
%

\section{Parallélisation de NTMIX}
\subsection{Décomposition de domaine}
\begin{frame}
  \begin{block}{Objectif}
    Exécuter NTMIX sur de grands maillages ($\approx 10^9$ points)
  \end{block}
  \pause
  \begin{alertblock}{Limitations matérielles}
    \begin{itemize}
    \item     Mémoire minimum nécessaire: $10^9pts \times 5 \times 8o \approx 37.25$ Go
    \item     Temps de calcul: $10^9 pts\times(4\times10^{-6})s/p\times10000it\approx463ans$
    \end{itemize}
  \end{alertblock}
  
\end{frame}


\begin{frame}

    \centering
    \visible<1>{\includegraphics[scale=0.15]{figures/globaldomain.png}}
    \visible<2>{\includegraphics[scale=0.15]{figures/partitionnedomain.png}}
\end{frame}




\begin{frame}
  Problème numérique

\footnotesize

\only<1>{$$3\left( \frac{\partial u}{\partial x}\right) _{i-1} + 9\left( \frac{\partial u}{\partial x}\right) _{i} + 3\left( \frac{\partial u}{\partial x}\right) _{i+1} = \frac{1}{h}\left(  \frac{1}{4} \left( u_{i+2}-u_{i-2} \right) + 7 \left( u_{i+1} - u_{i-1} \right) \right) $$}


\only<2>{$$3\left( \frac{\partial u}{\partial x}\right) _{i-1} + 9\left( \frac{\partial u}{\partial x}\right) _{i} + 3\left( \frac{\partial u}{\partial x}\right) _{i+1} = \frac{1}{h}\left(  \frac{1}{4} \left( \mathunderline{ u_{i+2}}-\mathunderline{u_{i-2}} \right) + 7 \left( \mathunderline{u_{i+1}} - \mathunderline{u_{i-1}} \right) \right) $$}

\only<3>{$$3\mathunderline{\left( \frac{\partial u}{\partial x}\right) _{i-1}} + 9\left( \frac{\partial u}{\partial x}\right) _{i} + 3\mathunderline{\left( \frac{\partial u}{\partial x}\right) _{i+1}} = \frac{1}{h}\left(  \frac{1}{4} \left( \mathunderline{ u_{i+2}}-\mathunderline{u_{i-2}} \right) + 7 \left( \mathunderline{u_{i+1}} - \mathunderline{u_{i-1}} \right) \right) $$}

\end{frame}


\begin{frame}
  Méthode Schwarz:
  \begin{itemize}
  \item Couplage inter-domaine élevé
  \item $\nearrow$ communications
  \item $\searrow$ taille des communications
  \end{itemize}
  Problème local modifié:
  \begin{itemize}
  \item Couplage inter-domaine faible
  \item $\searrow$ communications
  \item $\nearrow$ taille des communications
  \end{itemize}
\end{frame}


\subsection{Communications}
\begin{frame}

  \centering
    \only<1>{\includegraphics[scale=0.15]{slide_neigh_1.png}\\}
    \only<2>{\includegraphics[scale=0.15]{slide_neigh_2.png}\\}
    \only<3>{\includegraphics[scale=0.15]{slide_neigh_3.png}\\}
    \only<4>{\includegraphics[scale=0.15]{slide_neigh_4.png}}
\end{frame}

\begin{frame}
  \centering
    \only<1>{\includegraphics[scale=0.15]{slide_cust_1.png}\\}
    \only<2>{\includegraphics[scale=0.15]{slide_cust_2.png}\\}
    \only<3>{\includegraphics[scale=0.15]{slide_cust_3.png}\\}
    \only<4>{\includegraphics[scale=0.15]{slide_cust_4.png}}
\end{frame}




%
% Performances
%

\section{Etude des performances}
\subsection{Scalabilité forte}
\begin{frame}
  \begin{figure}[ht]
    \centering
    \includegraphics[page=2,scale=0.8]{gnuplot/bench_strong_nemo.pdf}
    \caption{\label{fig:label} }
  \end{figure}

\end{frame}


\subsection{Scalabilité faible}
\begin{frame}
  
\end{frame}

%
% Conclusion
%

\begin{frame}
  
\end{frame}

\end{document}
