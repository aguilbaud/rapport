\section*{Conclusion}\addcontentsline{toc}{section}{Conclusion}

\paragraph{}A l'issu de ce stage, une version séquentielle et tridimensionnelle de NTMIX est fonctionnelle. Pour cela, j'ai moderniser le code de simulation, développé en Fortran 77, afin qu'il puisse utiliser la mémoire dynamique, facilitant ainsi son utilisation. J'ai ensuite modifié l'application afin qu'elle puisse traiter des simulations tridimensionnelles au lieu de bidimensionnelles.


\paragraph{}J'ai également développé une version parallèle de cette application pouvant s'exécuter sur des clusters de calculs grâce à MPI. Cette version permettra de traiter de plus grands maillage que la version séquentielle, permettant ainsi d'améliorer la précision des simulation. Cependant, au moment de ou j'écris ce rapport, je n'ai pas encore pu mesurer les performances de la version parallèle de NTMIX dans des conditions réelles d'utilisation. Ce point ferra donc l'objet de la fin du stage.




\paragraph{}Cependant, il est encore possible d'améliorer les performances de NTMIX. Par exemple, la quantité de mémoire utilisée peut encore être réduite. En effet, NTMIX a été pensé avec une approche mémoire statique; même si je l'ai modifié pour qu'il puisse utiliser de la mémoire dynamique, certaines parties du code devraient être entièrement réécrites pour mieux correspondre à cette nouvelle approche. 
NTMIX réalise également de nombreuses opérations sur des matrices et il serait donc intéressant d'étudier ses performances avec l'utilisation de librairies \textit{BLAS}, mais ces modifications nécessiteraient, comme pour de nombreux "vieux" programme scientifique, de réécrire entièrement le code et ne pouvaient donc pas faire l'objet de ce stage.

